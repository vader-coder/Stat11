% Options for packages loaded elsewhere
\PassOptionsToPackage{unicode}{hyperref}
\PassOptionsToPackage{hyphens}{url}
%
\documentclass[
]{article}
\usepackage{lmodern}
\usepackage{amssymb,amsmath}
\usepackage{ifxetex,ifluatex}
\ifnum 0\ifxetex 1\fi\ifluatex 1\fi=0 % if pdftex
  \usepackage[T1]{fontenc}
  \usepackage[utf8]{inputenc}
  \usepackage{textcomp} % provide euro and other symbols
\else % if luatex or xetex
  \usepackage{unicode-math}
  \defaultfontfeatures{Scale=MatchLowercase}
  \defaultfontfeatures[\rmfamily]{Ligatures=TeX,Scale=1}
\fi
% Use upquote if available, for straight quotes in verbatim environments
\IfFileExists{upquote.sty}{\usepackage{upquote}}{}
\IfFileExists{microtype.sty}{% use microtype if available
  \usepackage[]{microtype}
  \UseMicrotypeSet[protrusion]{basicmath} % disable protrusion for tt fonts
}{}
\makeatletter
\@ifundefined{KOMAClassName}{% if non-KOMA class
  \IfFileExists{parskip.sty}{%
    \usepackage{parskip}
  }{% else
    \setlength{\parindent}{0pt}
    \setlength{\parskip}{6pt plus 2pt minus 1pt}}
}{% if KOMA class
  \KOMAoptions{parskip=half}}
\makeatother
\usepackage{xcolor}
\IfFileExists{xurl.sty}{\usepackage{xurl}}{} % add URL line breaks if available
\IfFileExists{bookmark.sty}{\usepackage{bookmark}}{\usepackage{hyperref}}
\hypersetup{
  pdftitle={hwk03},
  pdfauthor={Patrick Wheeler},
  hidelinks,
  pdfcreator={LaTeX via pandoc}}
\urlstyle{same} % disable monospaced font for URLs
\usepackage[margin=1in]{geometry}
\usepackage{color}
\usepackage{fancyvrb}
\newcommand{\VerbBar}{|}
\newcommand{\VERB}{\Verb[commandchars=\\\{\}]}
\DefineVerbatimEnvironment{Highlighting}{Verbatim}{commandchars=\\\{\}}
% Add ',fontsize=\small' for more characters per line
\usepackage{framed}
\definecolor{shadecolor}{RGB}{248,248,248}
\newenvironment{Shaded}{\begin{snugshade}}{\end{snugshade}}
\newcommand{\AlertTok}[1]{\textcolor[rgb]{0.94,0.16,0.16}{#1}}
\newcommand{\AnnotationTok}[1]{\textcolor[rgb]{0.56,0.35,0.01}{\textbf{\textit{#1}}}}
\newcommand{\AttributeTok}[1]{\textcolor[rgb]{0.77,0.63,0.00}{#1}}
\newcommand{\BaseNTok}[1]{\textcolor[rgb]{0.00,0.00,0.81}{#1}}
\newcommand{\BuiltInTok}[1]{#1}
\newcommand{\CharTok}[1]{\textcolor[rgb]{0.31,0.60,0.02}{#1}}
\newcommand{\CommentTok}[1]{\textcolor[rgb]{0.56,0.35,0.01}{\textit{#1}}}
\newcommand{\CommentVarTok}[1]{\textcolor[rgb]{0.56,0.35,0.01}{\textbf{\textit{#1}}}}
\newcommand{\ConstantTok}[1]{\textcolor[rgb]{0.00,0.00,0.00}{#1}}
\newcommand{\ControlFlowTok}[1]{\textcolor[rgb]{0.13,0.29,0.53}{\textbf{#1}}}
\newcommand{\DataTypeTok}[1]{\textcolor[rgb]{0.13,0.29,0.53}{#1}}
\newcommand{\DecValTok}[1]{\textcolor[rgb]{0.00,0.00,0.81}{#1}}
\newcommand{\DocumentationTok}[1]{\textcolor[rgb]{0.56,0.35,0.01}{\textbf{\textit{#1}}}}
\newcommand{\ErrorTok}[1]{\textcolor[rgb]{0.64,0.00,0.00}{\textbf{#1}}}
\newcommand{\ExtensionTok}[1]{#1}
\newcommand{\FloatTok}[1]{\textcolor[rgb]{0.00,0.00,0.81}{#1}}
\newcommand{\FunctionTok}[1]{\textcolor[rgb]{0.00,0.00,0.00}{#1}}
\newcommand{\ImportTok}[1]{#1}
\newcommand{\InformationTok}[1]{\textcolor[rgb]{0.56,0.35,0.01}{\textbf{\textit{#1}}}}
\newcommand{\KeywordTok}[1]{\textcolor[rgb]{0.13,0.29,0.53}{\textbf{#1}}}
\newcommand{\NormalTok}[1]{#1}
\newcommand{\OperatorTok}[1]{\textcolor[rgb]{0.81,0.36,0.00}{\textbf{#1}}}
\newcommand{\OtherTok}[1]{\textcolor[rgb]{0.56,0.35,0.01}{#1}}
\newcommand{\PreprocessorTok}[1]{\textcolor[rgb]{0.56,0.35,0.01}{\textit{#1}}}
\newcommand{\RegionMarkerTok}[1]{#1}
\newcommand{\SpecialCharTok}[1]{\textcolor[rgb]{0.00,0.00,0.00}{#1}}
\newcommand{\SpecialStringTok}[1]{\textcolor[rgb]{0.31,0.60,0.02}{#1}}
\newcommand{\StringTok}[1]{\textcolor[rgb]{0.31,0.60,0.02}{#1}}
\newcommand{\VariableTok}[1]{\textcolor[rgb]{0.00,0.00,0.00}{#1}}
\newcommand{\VerbatimStringTok}[1]{\textcolor[rgb]{0.31,0.60,0.02}{#1}}
\newcommand{\WarningTok}[1]{\textcolor[rgb]{0.56,0.35,0.01}{\textbf{\textit{#1}}}}
\usepackage{graphicx,grffile}
\makeatletter
\def\maxwidth{\ifdim\Gin@nat@width>\linewidth\linewidth\else\Gin@nat@width\fi}
\def\maxheight{\ifdim\Gin@nat@height>\textheight\textheight\else\Gin@nat@height\fi}
\makeatother
% Scale images if necessary, so that they will not overflow the page
% margins by default, and it is still possible to overwrite the defaults
% using explicit options in \includegraphics[width, height, ...]{}
\setkeys{Gin}{width=\maxwidth,height=\maxheight,keepaspectratio}
% Set default figure placement to htbp
\makeatletter
\def\fps@figure{htbp}
\makeatother
\setlength{\emergencystretch}{3em} % prevent overfull lines
\providecommand{\tightlist}{%
  \setlength{\itemsep}{0pt}\setlength{\parskip}{0pt}}
\setcounter{secnumdepth}{-\maxdimen} % remove section numbering

\title{hwk03}
\author{Patrick Wheeler}
\date{9/28/2020}

\begin{document}
\maketitle

\hypertarget{question-1}{%
\subsection{Question 1}\label{question-1}}

\#\#1. Create a scatterplot (remember to label axes and include a title)
with the results from the Shot put(I called this x \textless-
decathlon\$Shot.put) as the explanatory variable and the results from
the Discus (I called this y) as the response variable. Both are measured
in meters.

\begin{Shaded}
\begin{Highlighting}[]
\NormalTok{shotput <-}\StringTok{ }\NormalTok{decathlon}\OperatorTok{$}\NormalTok{Shot.put}
\NormalTok{discus <-}\StringTok{ }\NormalTok{decathlon}\OperatorTok{$}\NormalTok{Discus}
\NormalTok{plot1 <-}\StringTok{ }\KeywordTok{plot}\NormalTok{(discus}\OperatorTok{~}\NormalTok{shotput)}
\end{Highlighting}
\end{Shaded}

\includegraphics{hwk3_files/figure-latex/unnamed-chunk-1-1.pdf}

\begin{Shaded}
\begin{Highlighting}[]
\NormalTok{model <-}\StringTok{ }\KeywordTok{lm}\NormalTok{(discus}\OperatorTok{~}\NormalTok{shotput)}
\CommentTok{#abline(model)}
\end{Highlighting}
\end{Shaded}

\begin{enumerate}
\def\labelenumi{(\alph{enumi})}
\tightlist
\item
  Would it make sense if we switched the explanatory and response
  variables?\\
  Yes, because there is no difference between claiming success in
  discuss throwing correlates with success in shot putting and claiming
  success in shot putting correlates with success in discuss throwing.\\
\item
  Find the equation for the least squares regression line. Hint:
  functionlm(y\textasciitilde x). Briefly describe in words what the
  slope indicates.
\end{enumerate}

\begin{Shaded}
\begin{Highlighting}[]
\KeywordTok{lm}\NormalTok{(discus}\OperatorTok{~}\NormalTok{shotput)}
\end{Highlighting}
\end{Shaded}

\begin{verbatim}
## 
## Call:
## lm(formula = discus ~ shotput)
## 
## Coefficients:
## (Intercept)      shotput  
##       7.801        2.523
\end{verbatim}

The slope 2.523 indicates that the linear model predicts a ratio of
approximately 2.523 between the distance a contestant throws a discus
and the distance they shot put. It is unitless since both contests are
measured in units of length.\\
(c) In a new plot, add the line to your plot (Hint: use the function
abline() after your plot() function).

\begin{Shaded}
\begin{Highlighting}[]
\KeywordTok{plot}\NormalTok{(discus}\OperatorTok{~}\NormalTok{shotput)}
\KeywordTok{abline}\NormalTok{(model)}
\end{Highlighting}
\end{Shaded}

\includegraphics{hwk3_files/figure-latex/unnamed-chunk-3-1.pdf}

\begin{enumerate}
\def\labelenumi{(\alph{enumi})}
\setcounter{enumi}{3}
\tightlist
\item
  What is the mean-mean point, and is it on the line you just drew?\\
  The mean-mean point is the point (\(\Xbar\), \(\Ybar\)) where the
  coordinates are the mean of the explanatory variable \(\Xbar\) and the
  mean of the response variable \(\Ybar\). The linear model should
  include this point but let's test it to make sure:
\end{enumerate}

\begin{Shaded}
\begin{Highlighting}[]
\NormalTok{yBar =}\StringTok{ }\KeywordTok{c}\NormalTok{(}\KeywordTok{mean}\NormalTok{(discus))}\CommentTok{#mean of y: y-bar}
\NormalTok{xBar =}\StringTok{ }\KeywordTok{c}\NormalTok{(}\KeywordTok{mean}\NormalTok{(shotput))}\CommentTok{#mean of x: x-bar}
\NormalTok{meanMean =}\StringTok{ }\KeywordTok{data.frame}\NormalTok{(xBar, yBar)}
\KeywordTok{plot}\NormalTok{(discus}\OperatorTok{~}\NormalTok{shotput)}
\KeywordTok{abline}\NormalTok{(model)}
\KeywordTok{points}\NormalTok{(meanMean, }\DataTypeTok{col=}\StringTok{"red"}\NormalTok{, }\DataTypeTok{pch=}\DecValTok{19}\NormalTok{)}\CommentTok{#(x-bar, y-bar) will be red.}
\end{Highlighting}
\end{Shaded}

\includegraphics{hwk3_files/figure-latex/unnamed-chunk-4-1.pdf}

\begin{Shaded}
\begin{Highlighting}[]
\NormalTok{yBarPrediciton =}\StringTok{ }\KeywordTok{predict}\NormalTok{(model, xBar) }
\end{Highlighting}
\end{Shaded}

\(Ybar\) = 44.3256098\\
yBarPrediction = 45.2160175, 43.7779506, 45.064642, 43.7527213,
46.1242702, 43.9040968, 41.8100695, 42.5164884, 44.5600571, 44.1563892,
43.3490534, 39.79173, 41.759611, 49.0760918, 46.2251872, 47.9912343,
47.4866494, 44.3329939, 46.4270212, 44.7618911, 43.7779506, 45.1403297,
43.1472195, 45.897207, 45.064642, 45.4178514, 46.2504165, 44.2320769,
45.5692269, 42.8192393, 41.9866742, 40.7756705, 47.3857324, 46.0233533,
44.5600571, 42.1632789, 41.1541092, 41.3559431, 45.165559, 41.9362158,
45.4430807\\
Given that (\(\Xbar\),\(\Ybar\)) appears to be on the line when we plot
it and that the model returns the correct value for \(\Ybar\) when asked
to predict the response variable for \(\Xbar\), the linear model
includes the point.\\
(e) If an additional contestant threw the Shot put 13m, how far would
you expect them to throw theDiscus? Would you be surprised if they in
fact threw 1m more? 10m more? Briefly explain.

\end{document}
